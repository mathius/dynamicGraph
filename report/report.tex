\documentclass[11pt, a4paper,draft]{article}
\usepackage{lmodern} % newer fonts
\usepackage[T1]{fontenc} % to make text copyable, before inputenca

\usepackage[utf8]{inputenc}
\usepackage[czech]{babel}
\usepackage{a4wide}
\usepackage[fleqn]{amsmath}  %matematické rovnice
\usepackage{amssymb}  %dvojitý fonty...
\usepackage{amsthm}   %důkazy
%\usepackage{latexsym}   %\box
\usepackage{IEEEtrantools} %lepší bloky rovnic
\usepackage{tabulary} %tabulky

\usepackage{xcolor}
\definecolor{dark-red}{rgb}{0.6,0.15,0.15}
\definecolor{dark-green}{rgb}{0.15,0.4,0.15}
\definecolor{medium-blue}{rgb}{0,0,0.5}
\usepackage[ plainpages=false
           , pdfpagelabels
           , unicode
           , draft=false
           , colorlinks=true
           , linkcolor={dark-red}
           , citecolor={dark-green}
           , urlcolor={medium-blue}
           ]{hyperref}


%\addtolength{\headsep}{-0.5cm}
%\addtolength{\textheight}{0.5cm}
%  \addtolength{\hoffset}{-1.6cm}
%  \addtolength{\textwidth}{1cm}

\usepackage{fancyhdr} %hlavičky
  \pagestyle{fancy}
  \fancyhead[RO]{Andrej Krejčír, Vladimír Štill, Martin Ukrop}
  \fancyhead[LO]{IB013 Assignment report}
  
\newcommand{\N}{\mathbb{N}}
\newcommand{\Z}{\mathbb{Z}}
\newcommand{\Q}{\mathbb{Q}}
\newcommand{\R}{\mathbb{R}}
\newcommand{\bsl}{\backslash}
\newcommand{\interval}[2]{\ensuremath{\left(#1, #2\right)}}
\newcommand{\id}{\,\mathrm{d}}
\newcommand{\set}[1]{\left\{#1\right\}}
\newcommand{\pro}{\text{pro }}

\newcommand{\pl}[1]{\texttt{#1}} % maybe there is better way -- listing??

\newcommand{\Oh}[1]{\ensuremath{\mathcal{O}\left(#1\right)}}
%\usepackage{graphicx}
\usepackage{csquotes}
% Nastaveni spravnych ceskych uvozovek
\renewcommand{\uv}[1]{\quotedblbase #1\textquotedblleft}
\newcommand{\uvv}[1]{\quotesinglbase #1\textquoteleft}

\usepackage{algorithm}
\usepackage{algpseudocode}
\usepackage{float}
\usepackage{textcomp}
\usepackage{listings}

% adjust line number margin
\newlength{\MaxSizeOfLineNumbers}%
\settowidth{\MaxSizeOfLineNumbers}{\scriptsize 99:\ \,}% Adjust to maximum number of lines

\lstset{ language=Prolog
       , numberstyle=\scriptsize
       , numbers=left
       }

\makeatletter
\def\lst@PlaceNumber{%
    \makebox[\MaxSizeOfLineNumbers][r]{\normalfont\lst@numberstyle{\thelstnumber:\ \,}}%
}
\makeatother

\usepackage{verbatim}

\theoremstyle{plain}
\newtheorem{lemma}{Lemma}[section]
\newtheorem{conseq}[lemma]{Důsledek}
\theoremstyle{definition}
\newtheorem{definition}[lemma]{Definice}
\theoremstyle{remark}
\newtheorem{remark}[lemma]{Poznámka}


\begin{document}
\section{USER INTERFACE}

\begin{itemize}
    \item v casovom momente
    \item v intervale
    \item pocas celej existencie grafu

    \item zapnut generovanie grafov
        \begin{itemize}
             \item 1 parameter s prefixom nazvu suboru
             \item vypisovat kolko minut bude este hrana existovat
         \end{itemize}
    \item vypnut generovanie grafov

    \item analyzovat, v ktorych okamihoch nastala zmena
        \begin{itemize}
            \item ak graph, vyegenerovat \texttt{.dot} pre kazdy casovy okamih v danom intervale, kde nastala zmena
             \item varovanie ak sa vygeneruje viac ako 10 suborov
         \end{itemize}
    \item najst najvacsiu komponentu a prvy casovy okamih, v ktorom existuje
        \begin{itemize}
              \item vypisat jej velkost, vrcholy a hrany
              \item ak graph, vygeneruje .dot v prvom okamihu existencie tej najvacsej komponenty, najvacsia komponenta bude odlisena farebne
          \end{itemize}
    \item analyzovat vrchol (vsetky hrany existujuce aspon v 1 momente v zadanom casovom intervale)
    \item vypisat prehlad komponent zaciatocnom casovom momente
        \begin{itemize}
           \item ak graph, vygenerovat dot
        \end{itemize}
    \item statistika vrcholov
        \begin{itemize}
          \item vrchol s najvacsim/najmensim poctom hran
          \item yvpise sa vrchol, pocet E hran
      \end{itemize}
    \item statistika hran
        \begin{itemize}
          \item hrana s najdlhsim/najkratsim(trvanie >=1 ) trvanim
          \item vypisaju sa vrcholy, trvanie od-do
        \end{itemize}
\end{itemize}

\section{FORMAT SUBOROV}

\subsection{format grafu}
\begin{itemize}
    \item synax, syntax timestamp
    \item vrcholy \texttt{a-zA-Z0-9} stringy zacinajuce malym pismenom
    \item bez hran a->a
    \item trvanie hran je >=0
    \item neE 2 hrany medzi a,b v 1 okamihu
\end{itemize}

\section{GENERATOR DAT}

\subsection{vstup}
\begin{itemize}
    \item počet vrcholů -- \pl{vertices( V )}.
        \begin{itemize}
            \item V > 0
        \end{itemize}
    \item interval počtu hran \pl{edges( Min, Max )} v každém časovém okamžiku
        \begin{itemize}
            \item musí platit $\pl{Min} \le \pl{Max} \land Max \le \frac{ V(V-1)}{2}$
        \end{itemize}
    \item počet vrcholů
    \item predikát \pl{newEdge( Time, Probability )}, který pro daný čas vrací
          pravděpodobnost vzniku hrany mezi libovonlými nespojenými vrcholy
        \begin{itemize}
            \item musí platit $\pl{Min} \le \pl{Max}$
        \end{itemize}
    \item predikát \pl{removeEdge( Time, Duration, Probability )}, který pro daný čas
    a dobu aktuálního trvání hrany, vrací pravděpodobnost, že hrana v tomto
    časovém okamžiku zanikne
    \item musí platit $0 \le \pl{Probability} \le 1$, floating point
    \item doba trvání grafu \pl{duration( BeginTime, EndTime )}
        \begin{itemize}
            \item $\pl{BeginTime} \le \pl{EndTime}$
            \item \texttt{BeginTime, EndTime :: DateTime}
        \end{itemize}
\end{itemize}

\subsection{graph generation}
\begin{algorithmic}[1]
    \State $Q_A \leftarrow$ queue of all possible edges (without time -- pairs of vertices)
    \ForAll { minute $T \in \langle BeginTime, EndTime \rangle $ }
        \ForAll { edge $E \in Graph( T )$ }
            \State call $removeEdge$ with current time $T$ and current duration of $E$
            \State remove $E$ with $Probability$
            \If { removed } push to $Q_A$ \EndIf
        \EndFor
        \State call $newEdge( T, Probability )$.
        \State $Q_a \leftarrow empty queue$
        \While { not empty $_A$ }
            \If { $current edge count == Max$ }
                \State push all edges from $Q_a$ to $Q_A$
                \State continue to next minute
            \EndIf
            \State $E = pop( Q_A )$
            \State with $Probability$ add $E$ to graph
            \If { not added } push add E to $Q_a$ \EndIf
            \If $Q_A$ is empty and $current edge count < Min$
                \While { current edge count < Min }
                    \State $E = pop( Q_a )$
                    \State with $Probability$ add $E$ to graph
                    \If { not added } push $E$ to $Q_a$ \EndIf
                \EndWhile
            \EndIf
        \EndWhile
        \State $swap( Q_A, Q_a )$ \Comment{ $Q_A$ was empty}
    \EndFor
\end{algorithmic}

\subsection{Formát vstupu}
\begin{itemize}
    \item vstup od uživatele na příkaz \pl{create\_graph( Filename )}.
    \item dávkový stup ze souboru
        \begin{itemize}
            \item soubor může obsahovat více deklarací grafu, pak jsou generovány postupně
            \item deklarace začíná klauzulí \pl{create\_graph( Filename ).}
            \item pokračuje až po další klauzuli \pl{create\_graph( Filename2 ).}, nebo
                \pl{po end\_of\_file.}
        \end{itemize}
    \item příslušný typ generátoru je automaticky detekován.
\end{itemize}

\subsection{Práce se vstupem:}
\begin{itemize}
    \item pokud generátor očekává na dané pozici výstupní číselnou hodnotu, je nutné
    aby hodnota byla aritmeticky vyhodnotitelná (tedy se může jednat o číslo,
    nebo aritmetický výraz -- viz \pl{edges/2} v příkladech).
    \item v přípdě, že vstup nebude korektní, tedy bude například kombinovat predikáty
    z různých generátorů, generátor ohlásí chybu a odmítne generovat
    \item v případě, že je dodán nekorektní vstup a je tak zjištěno po začátku generování
    (rozšířený generátor), generátor vypíše varování a ukončí generování v daném
    okamžiku (vrátí dosud vygenerovaný graf).
    \item obdobně, v případě, že selže některý z uživatelem dodaných predikátů generátor
    vypíše varování a korektně ukončí generování v daném okamžiku
    \item v případě dávkového vstupu se chyby řeší pro každý graf jednotlivě
\end{itemize}

\subsection{náhodná čísla}
\begin{itemize}
    \item library( random )
    \item načtení: use\_module( library( random ) ).
    \item \url{https://www.fi.muni.cz/~hanka/sicstus/doc/html/sicstus/lib_002drandom.html#lib_002drandom}
\end{itemize}

\subsection{ukázkový vstup (možná podoba)}

\subsubsection{Interaktivní režim}
(za > je vstup uživatele)
\begin{verbatim}
> dot.
> graphGenerate( 'test01' ).
vertices( N ).> 16
edges( Min, Max ).> 8, 150
newEdge( Probability ).> 0.5
removeEdge( Probability ).> 0.5
duration( BeginTime, EndTime ).> date( 2012, 4, 5, 14, 00 ), date( 2012, 4, 7, 22, 00 )
generating 'test01.graph' ... done.
generating 'test01.dot' ... done.

>
\end{verbatim}

\subsubsection{Dávkový režim}

\begin{lstlisting}[firstnumber=1]
% gen_graph.pl
graphGenerate( 'test01.graph', 'test01.dot' ).
vertices( 16 ).
edges( 8, 150 ).
newEdge( _, 0.5 ).
removeEdge( _, _, 0.5 ).
duration( date( 2012, 4, 5, 14, 00 ), date( 2012, 4, 7, 22, 00 ) ).

graphGenerate( 'test02.graph' ).
vertices( 256 ).
edges( 0, ( 256 * 255 ) / 2 ).
newEdge( 0, 0.9 ).
newEdge( Time, 0.8 ) :- Time mod 60 =:= 0, !.
newEdge( _Time, 0.1 ).
removeEdge( _Time, Duration, Probability ) :-
    Duration >= 120,
    !,
    Probability = 1.
removeEdge( _Time, Duration, Probability ) :-
    Probability is duration / 120.
duration( date( 1970, 1, 1, 0, 0 ), date( 2038, 1, 19, 3, 14 ) ).

end_of_file.
\end{lstlisting}

\section{Examples}
\verbatiminput{session.txt}


\end{document}
