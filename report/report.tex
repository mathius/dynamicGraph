\documentclass[11pt, a4paper,draft]{article}
\usepackage{lmodern} % newer fonts
\usepackage[T1]{fontenc} % to make text copyable, before inputenca

\usepackage[utf8]{inputenc}
\usepackage[english]{babel}
\usepackage{a4wide}
\usepackage[fleqn]{amsmath}  %matematické rovnice
\usepackage{amssymb}  %dvojitý fonty...
\usepackage{amsthm}   %důkazy
%\usepackage{latexsym}   %\box
\usepackage{IEEEtrantools} %lepší bloky rovnic
\usepackage{tabulary} %tabulky

\usepackage{xcolor}
\definecolor{dark-red}{rgb}{0.6,0.15,0.15}
\definecolor{dark-green}{rgb}{0.15,0.4,0.15}
\definecolor{medium-blue}{rgb}{0,0,0.5}
\usepackage[ plainpages=false
           , pdfpagelabels
           , unicode
           , draft=false
           , colorlinks=true
           , linkcolor={dark-red}
           , citecolor={dark-green}
           , urlcolor={medium-blue}
           ]{hyperref}


%\addtolength{\headsep}{-0.5cm}
%\addtolength{\textheight}{0.5cm}
%  \addtolength{\hoffset}{-1.6cm}
%  \addtolength{\textwidth}{1cm}

\usepackage{fancyhdr} %hlavičky
  \pagestyle{fancy}
  \fancyhead[RO]{Andrej Krejčír, Vladimír Štill, Martin Ukrop}
  \fancyhead[LO]{IB013 Assignment report}
  
\newcommand{\N}{\mathbb{N}}
\newcommand{\Z}{\mathbb{Z}}
\newcommand{\Q}{\mathbb{Q}}
\newcommand{\R}{\mathbb{R}}
\newcommand{\bsl}{\backslash}
\newcommand{\interval}[2]{\ensuremath{\left(#1, #2\right)}}
\newcommand{\id}{\,\mathrm{d}}
\newcommand{\set}[1]{\left\{#1\right\}}
\newcommand{\pro}{\text{pro }}

\newcommand{\pl}[1]{\texttt{#1}} % maybe there is better way -- listing??

\newcommand{\Oh}[1]{\ensuremath{\mathcal{O}\left(#1\right)}}
%\usepackage{graphicx}
\usepackage{csquotes}
% Nastaveni spravnych ceskych uvozovek
%\renewcommand{\uv}[1]{\quotedblbase #1\textquotedblleft}
%\newcommand{\uvv}[1]{\quotesinglbase #1\textquoteleft}

\usepackage{algorithm}
\usepackage{algpseudocode}
\usepackage{float}
\usepackage{textcomp}
\usepackage[final]{listings}

% adjust line number margin
\newlength{\MaxSizeOfLineNumbers}%
\settowidth{\MaxSizeOfLineNumbers}{\scriptsize 99:\ \,}% Adjust to maximum number of lines

\lstset{ language=Prolog
       , numberstyle=\scriptsize
       , numbers=left
       }

\makeatletter
\def\lst@PlaceNumber{%
    \makebox[\MaxSizeOfLineNumbers][r]{\normalfont\lst@numberstyle{\thelstnumber:\ \,}}%
}
\makeatother

\usepackage{verbatim}

\theoremstyle{plain}
\newtheorem{lemma}{Lemma}[section]
\newtheorem{conseq}[lemma]{Důsledek}
\theoremstyle{definition}
\newtheorem{definition}[lemma]{Definice}
\theoremstyle{remark}
\newtheorem{remark}[lemma]{Poznámka}

\newcommand{\qq}[1]{\textit{\textcolor{red}{#1}}}
\newcommand{\gv}{\textit{Graphviz}}

\begin{document}
\title{{\Huge DynamicGraph} \\ {\large mid-term report for IB013 Logic Programming} }
\author{Andrej Krejčír, Vladimír Štill, Martin Ukrop}
\date{\today}
\maketitle

\section{User interface}

DynamicGraph provides interactive command line interface. After running the compiled binary (or calling the
top level predicate \pl{dynamicGraph/0}), basic program info is displayed followed by the internal command prompt (\pl{|: }).
 All the commands are using Prolog predicate syntax - i. e. they must always end with a dot character (\pl{.}) 
 with optional parameters in parentheses before it. Commands are divided into several categories.

The interactive interface holds several global settings that can be altered by specific commands. 
These settings include currently loaded graph, boundaries of the time interval and status of the \gv{} support. 
Settings are stored as facts dynamically added to program database.

\subsection{Graph manipulating commands}
\begin{itemize}
\item \pl{graphLoad/1}, \pl{gl/1} \\
Load graph from specified file. Syntax of the file contents is checked, if it is not correct, file is not loaded.

\item \pl{graphGenerate/1}, \pl{gg/1} \\
Generate new graph according to settings in specified file. 
Syntax of the configuration file is checked, if incorrect, the graph is not generated. 
After generation, graph is automatically saved to file \pl{<graphname>.pl} to current folder.

\item \pl{graphGenerate/0}, \pl{gg/0} \\
Generate new graph, parameters are set interactively. 
Syntax  of the values from user are checked, if incorrect, the graph is not generated. 
After generation, graph is automatically saved to file \pl{<graphname>.pl} to current folder.

\item \pl{graph/0}, \pl{g/0} \\
Display info about currently loaded graph - name, number or nodes and number of edges.
\end{itemize}

\subsection{Time commands}

\begin{itemize}
\item \pl{timeBegin/1}, \pl{tb/1} \\
Set beginning of time interval. 
Time is provided in time stamp format. 
Input format is checked.

\item \pl{timeBegin/0}, \pl{tb/0} \\
Set first edge occurence as beginning of time interval.

\item \pl{timeEnd/1}, \pl{te/1} \\
Set end of time interval. 
Time is provided in time stamp format. 
Input format is checked.

\item \pl{timeEnd/0}, \pl{te/0} \\
Set last edge occurence as beginning of time interval.

\item \pl{timeInterval/2}, \pl{ti/2} \\
Set beginning/end of time interval. 
Equivalent to setting beginning and end separately. 
Time is provided in time stamp format. 
Input format is checked.

\item \pl{timeInterval/0}, \pl{ti/0} \\
Set time interval from first to last edge occurrence. 
Equivalent to using \pl{timeBegin/0} with \pl{timeEnd/0}.

\item \pl{timeMoment/1}, \pl{tm/1} \\
Set both beginning/end of time interval to the same value. 
Equivalent to using \pl{timeBegin/1} and \pl{timeEnd/1} with the same time stamp.

\item \pl{time/0}, \pl{t/0} \\
Display currently set time interval (beginning, end).
\end{itemize}

\subsection{Graphviz output commands}

\begin{itemize}
\item \pl{graphviz/1}, \pl{gv/1} \\
Enable \gv{} outputs to specified folder. 
Folder existance is not checked. 
All commands providing \gv{} functionality will output to this folder. 
To change the output folder, run \pl{graphviz/1} again with new parameter.

\item \pl{graphvizOff/0}, \pl{gvo/0} \\
Disable \gv{} outputs. No following commands will create any \gv{} files.
\end{itemize}

\subsection{Graph statistics commands}

\begin{itemize}
\item \pl{statsNodes/0}, \pl{sn/0} \\
Display statistics for nodes. 
For the set time interval, the node with the most and the least (zero included) edges is determined. 
Displayed is the node name and edge count. 
In case of multiple results, only the first is displayed. \\
\textit{Note: Edges existing in any point of the time interval are considered.}

\item \pl{statsEdges/0}, \pl{se/0} \\
Display statistics for edges. 
For the set time interval, the edge with the longest and the shortest (but non-zero) duration is determined.  
Displayed is the edge beginning node, end node and time interval. 
In case of multiple results, only the first is displayed.

\item \pl{statsComponents/0}, \pl{sc/0} \\
Display the components at the beginning of the time interval. 
For each component, its nodes and node count are displayed. 
If \gv{} support is enabled, graph file for this moment is generated and saved. 
Graph file name consists of graph name and selected time moment.

\item \pl{statsProgress/0}, \pl{sp/0} \\
Display times when graph is changing (edge is added or removed). 
If \gv{} support is enabled, graph file for each of these moments is generated. 
Graph file names consists of graph name and time moment of the change. \\
\textit{Note: As a precaution, if the number of files is greater than 10, confirmation is required from the user.}

\item \pl{statsAnalyseNode/1}, \pl{san/1} \\
Analyse given node specified by its name. 
All edges to this node in the selected time interval are displayed along with their count.

\item \pl{statsMaxComponent/0}, \pl{smc/0} \\
Determines the maximal component from all the time moments in the selected interval. 
For the result component, its nodes, edges and size is displayed. 
In case of multiple results, only the first is displayed.
\end{itemize}

\subsection{Miscellaneous other commands}

\begin{itemize}
\item \pl{help/0}, \pl{h/0} \\
Display the list of commands with short description.

\item \pl{quit/0}, \pl{q/0} \\
Quit DynamicGraph session.
\end{itemize}

\section{File specification}
\label{sec:files}

\subsection{Graph file}
\label{sec:files-graph}

The input file contains definitions of edges separated by whitespace.
Edge is defined as this prolog predicate: \pl{e(From, To, Timestamp, Duration)}.
Only one graph can be defined in one file.

\begin{itemize}
	\item \pl{From} and \pl{To} are names of nodes, using only characters \texttt{a-z A-Z 0-9}
		and starting with s lowercase latter.	
		Loop edges are not supported: \pl{From} $\ne$ \pl{To}.
	\item \pl{Timestamp} is the time when the edge is created,
		with this format: \pl{YYYY-MM-DD+HH:MM}.
	\item \pl{Duration} is the number of minutes, how long the edge exists, must be nonnegative.
		Zero value means that only nodes are created, no edge between them.
		There can't be multiple edges between two nodes at the same time.
\end{itemize}
Example of a valid input file:

\begin{verbatim}
e(a, b, 2013-04-08+12:00, 60)
e(b, c, 2013-04-08+10:00, 200)
e(c, b, 2013-04-07+22:24, 3000)

% This edge is valid, it starts right after the first ends
e(a, b, 2013-04-08+13:00, 60)

e(e, f, 2013-04-08+20:00, 0)

\end{verbatim}

\subsection{Graph generator configuration file}
\label{sec:files-generator}

\begin{itemize}
    \item The declaration of graph to be generated starts with \pl{graphGenerate( Filename ).}
    \item The declaration ends by next occurence of \pl{graphGenerate/1} or by
        \pl{end\_of\_file.}
    \item Input file can contain multiple graph definitions.
    \item Input file is valid Prolog file but it is not intended to be consulted or
        compiled with DynamicGraph as a library due to the fact it can contain multiple
        graphs.
    \item Custom predicates can be internally used in configuration file, their scope
        is limited to one graph declaration.
    \item Graph is described by following predicates (all mandatory): 
        \begin{itemize}
            \item \pl{vertices/1}

                Number of vertices of graph. For \pl{vertices( V ).}, it must hold that $V > 0$.
            \item \pl{edges/2}

                Minimal and maximal number of edges at any point in time.
                For \pl{edges( Min, Max ).}, it must hold that
                $\pl{Min} \le \pl{Max} \land Max \le \frac{ V(V-1)}{2}$.
            \item \pl{newEdge/2}

                \pl{newEdge( Time, Probability )} gives probability \pl{Probability} of
                edge creation at given time \pl{Time} for all possible edges not currently
                included in graph. It must hold that $\pl{Min} \le \pl{Max}$.
            \item \pl{removeEdge/3}
                
                \pl{removeEdge( Time, Duration, Probability ).} for given time \pl{Time}
                and given duration \pl{Duration} of edge at \pl{Time} gives
                probability \pl{Probability} of edge removal (that is probability
                that given edge will have its current duration).
                It must hold that $0 \le \pl{Probability} \le 1$.
            \item \pl{duration/2}

                Time interval in which graph exists. For \pl{duration( BeginTime, EndTime )}
                it must hold that  $\pl{BeginTime} \le \pl{EndTime}$ and both
                \pl{BeginTime} and \pl{EndTime} are valid timestamps.
        \end{itemize}

\end{itemize}
Example of the valid graph generator configuration file follows.

\begin{verbatim}
% gen_graph.pl
graphGenerate( 'test01.graph', 'test01.dot' ).
vertices( 16 ).
edges( 8, 150 ).
newEdge( _, 0.5 ).
removeEdge( _, _, 0.5 ).
duration( 2012-4-5+14:00, 2012-4-7+22:00 ).

graphGenerate( 'test02.graph' ).
vertices( 256 ).
edges( 0, ( 256 * 255 ) / 2 ).
newEdge( 0, 0.9 ).
newEdge( Time, 0.8 ) :- Time mod 60 =:= 0, !.
newEdge( _Time, 0.1 ).
removeEdge( _Time, Duration, Probability ) :-
    Duration >= 120,
    !,
    Probability = 1.
removeEdge( _Time, Duration, Probability ) :-
    Probability is duration / 120.
duration( 1970-1-1+00:00, 2038-1-19+3:14 ).

end_of_file.
\end{verbatim}

\subsection{Graphviz files}
\label{sec:files-graphviz}

DynamicGraph provides graphical output using the \gv{} library. 
If this output is enabled (predicate \pl{graphviz/1}), DynamicGraph generated \pl{.dot} files, 
that can be displayed using the \textit{Dot viewer} from the \gv{} library. 
For further information of \gv{} and full syntax of \pl{.dot} files, consult \url{http://www.graphviz.org/}.

\section{Computing graph statistics}
\label{sec:computing}


%Solution analysis. To be added.

\subsection{Time representation}
\label{sec:computing-time}

The input time format \pl{YYYY-MM-DD+HH:MM} is internally converted to a single number,
representing the number of minutes passed since \pl{2000-01-01+00:00}. 
Similar to UNIX time. Time before this date is represented by a negative number.
 

\subsection{Graph representation}
\label{sec:computing-graph} 

The loaded graph, is saved to dynamic database. This means that there can be only one graph loaded at the same time. Before a new graph is loaded the old one is deleted form database.

Edges are inserted into dynamic database as facts \pl{edge(From,To,Start,End).}\\
Nodes can be represented as this predicate: \\
\pl{ node(X) :- edge(X,\_,\_,\_). \\
 node(X) :- edge(\_,X,\_,\_).}
 
\subsection{Random numbers}
\label{sec:computing-random}
\begin{itemize}
    \item library( random )
    \item use\_module( library( random ) ).
    \item \url{https://www.fi.muni.cz/~hanka/sicstus/doc/html/sicstus/lib_002drandom.html#lib_002drandom}
\end{itemize}

\section{Graph generator}
\label{sec:generator}

\subsection{Graph generator settings}
\label{sec:generator-settings}

Graph generator requires the following parameters to be set:
\begin{itemize}
    \item number of vertices, $V > 0$
    \item minimal and maximal number of edges at any moment in time, 
        $\pl{Min} \le \pl{Max} \land Max \le \frac{ V(V-1)}{2}$
    \item probability of edge apperance in given time
    \item probability of edge removal given time and current edge duration
    \item time interval in which graph exists
\end{itemize}

For more information about generator invocation see following section.

\subsection{Graph generator modes}
\label{sec:generator-modes}
Settings for graph generator can be set in different 2 modes:
\begin{itemize}
\item \textbf{Interactive mode} \\
This mode is activated when using \pl{graphGenerate/0}. 
User is prompted for all settings one-by-one. Input values are checked for correct syntax. 
For an example of interactive graph generation, see \autoref{sec:exmaple-cli}.
\item \textbf{Batch mode} \\
This mode is activated when using \pl{graphGenerate/1}. 
Settings are loaded from external file specified in parameter. 
The file syntax is checked -- if incorrect, the user is informed and the graph is not generated. 
For detailed syntax of this file, see \autoref{sec:files-generator}.
\end{itemize}

Any time generator expects numeric value, any arithmetical expression
that is arithmetically solvable by Prolog can be used.

If input inconsistency is detected at runtime warning is emitted and generator
continues with next graph.

\subsection{Graph generation}
\label{sec:generator-generation}

\begin{algorithmic}[1]
    \State $Q_A \leftarrow$ queue of all possible edges (without time -- pairs of vertices)
    \ForAll { minute $T \in \langle BeginTime, EndTime \rangle $ }
        \ForAll { edge $E \in Graph( T )$ }
            \State call $removeEdge$ with current time $T$ and current duration of $E$
            \State remove $E$ with $Probability$
            \If { removed } push to $Q_A$ \EndIf
        \EndFor
        \State call $newEdge( T, Probability )$.
        \State $Q_a \leftarrow empty queue$
        \While { not empty $_A$ }
            \If { $current edge count == Max$ }
                \State push all edges from $Q_a$ to $Q_A$
                \State continue to next minute
            \EndIf
            \State $E = pop( Q_A )$
            \State with $Probability$ add $E$ to graph
            \If { not added } push add E to $Q_a$ \EndIf
            \If $Q_A$ is empty and $current edge count < Min$
                \While { current edge count < Min }
                    \State $E = pop( Q_a )$
                    \State with $Probability$ add $E$ to graph
                    \If { not added } push $E$ to $Q_a$ \EndIf
                \EndWhile
            \EndIf
        \EndWhile
        \State $swap( Q_A, Q_a )$ \Comment{ $Q_A$ was empty}
    \EndFor
\end{algorithmic}

\section{Examples}
\section{Example CLI session}
\label{sec:exmaple-cli}
\verbatiminput{session.txt}


\end{document}
